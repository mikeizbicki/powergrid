\begin{frame}{Naive solution: joint estimation with the UKF}

Augment our system dynamics
\begin{equation*}
\x_{t+1} = (A+BA^p)\x_t + B(\uu_t^n+\uu_t^p) + \epsilon
%\x_{t+1} = (\timestep A+sBA^p+I)\x_t + \timestep B(\uu_t^n+\uu_t^p) + \epsilon
\end{equation*}
to include the new state variables
\begin{equation*}
\label{eq:aug}
\begin{aligned}
\begin{split}
\begin{bmatrix}
\x_{t+1} \\
\vectorize K^{LG}_{t+1}
\end{bmatrix}
&=
\begin{bmatrix}
A + BA^p_t & 0 \\
%sA + sBA^p_t + I & 0 \\
0 & I
\end{bmatrix}
\begin{bmatrix}
\x_t \\
\vectorize K^{LG}_t
\end{bmatrix}
%\\&~~~~~~~~~~+
+
\begin{bmatrix}
B & 0\\
%sB & 0\\
0 & I
\end{bmatrix}
\begin{bmatrix}
\uu_t^n + \uu_t^p \\
\uu^{LG}_t
\end{bmatrix}
+
\begin{bmatrix}
\epsilon \\
\epsilon^{LG}
\end{bmatrix}
\end{split}
.
\end{aligned}
\end{equation*}

The resulting system is nonlinear.

We solve it with the Unscented Kalman Filter (UKF).

\noindent\rule[0.5ex]{\linewidth}{1pt}
\pause

Note that:
\begin{itemize}
\item $\x$ has $O(\ell+g)$ components (size of original problem)
\item $K^{LG}$ has $O(\ell g)$ components (size of new problem)
\end{itemize}

\pause
Disadvantages of new problem:
\begin{itemize}
\item Computational cost is $O((\ell g)^3)$, much worse than $O((\ell+g)^3)$
\item Not enough data for statistical efficiency
\end{itemize}
\end{frame}

%%%%%%%%%%%%%%%%%%%%%%%%%%%%%%%%%%%%%%%%%%%%%%%%%%%%%%%%%%%%%%%%%%%%%%%%%%%%%%%%

\begin{frame}{Our solution: joint estimation with rank-1 UKF}
Assume that $K^{LG}$ has rank 1:
\begin{equation*}
K^{LG}_t=\kL_t\trans{\kG}_{t}.
\end{equation*}
This is reasonable because there will typically be few attacks.

Then the system dynamics become
\begin{equation*}
\begin{bmatrix}
\x_{t+1} \\
\kL_{t+1} \\
\kG_{t+1} \\
\end{bmatrix}
=
\begin{bmatrix}
A + BA^p_t & 0 & 0\\
%sA + sBA^p_t + I & 0 & 0\\
0 & I & 0\\
0 & 0 & I
\end{bmatrix}
\begin{bmatrix}
\x_t \\
\kL_{t} \\
\kG_{t} \\
\end{bmatrix}
+
\begin{bmatrix}
B & 0 & 0\\
%sB & 0 & 0\\
0 & I & 0\\
0 & 0 & I\\
\end{bmatrix}
\begin{bmatrix}
\uu_t^n + \uu_t^p \\
\uu^L_t \\
\uu^G_t
\end{bmatrix}
+
\begin{bmatrix}
\epsilon \\
\epsilon^K \\
\epsilon^L
\end{bmatrix}
.
\end{equation*}
Solve the system using the UKF.

\noindent\rule[0.5ex]{\linewidth}{1pt}
\pause

Note that:
\begin{itemize}
\item The size of the new state space is $O(\ell+g)$
\item Computationally and statistically efficient
\end{itemize}

\end{frame}
