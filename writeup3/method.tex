\documentclass{article}
\usepackage{graphicx}
\usepackage{geometry}
\usepackage{amsmath}

%%%%%%%%%%%%%%%%%%%%%%%%%%%%%%%%%%%%%%%%

\DeclareMathOperator*{\vectorize}{vec}

\newcommand{\normal}[2]{\ensuremath{\mathcal{N}\left({{#1}},{{#2}}\right)}}
\newcommand{\trans}[1]{\ensuremath{{#1}^{\mathsf{T}}}}

\newcommand{\e}{\mathbf{e}}
\newcommand{\bb}{\mathbf{b}}
\newcommand{\w}{\mathbf{w}}
\newcommand{\x}{\mathbf{x}}
\newcommand{\y}{\mathbf{y}}
\newcommand{\uu}{\mathbf{u}}
\newcommand{\vv}{\mathbf{v}}

%%%%%%%%%%%%%%%%%%%%%%%%%%%%%%%%%%%%%%%%%%%%%%%%%%%%%%%%%%%%%%%%%%%%%%%%%%%%%%%%

\begin{filecontents}{paper.bib}
@inproceedings{wan2000unscented,
  title={The unscented Kalman filter for nonlinear estimation},
  author={Wan, Eric A and Van Der Merwe, Rudolph},
  booktitle={Adaptive Systems for Signal Processing, Communications, and Control Symposium 2000. AS-SPCC. The IEEE 2000},
  pages={153--158},
  year={2000},
  organization={Ieee}
}

@inproceedings{morelande2006reduced,
  title={Reduced sigma point filtering for partially linear models},
  author={Morelande, Mark R and Ristic, Branko},
  booktitle={2006 IEEE International Conference on Acoustics Speech and Signal Processing Proceedings},
  volume={3},
  pages={III--III},
  year={2006},
  organization={IEEE}
}

@inproceedings{padilla2010adaptive,
  title={An adaptive-covariance-rank algorithm for the unscented Kalman filter},
  author={Padilla, Lauren E and Rowley, Clarence W},
  booktitle={49th IEEE Conference on Decision and Control (CDC)},
  pages={1324--1329},
  year={2010},
  organization={IEEE}
}

@article{wang2010generating,
  title={Generating statistically correct random topologies for testing smart grid communication and control networks},
  author={Wang, Zhifang and Scaglione, Anna and Thomas, Robert J},
  journal={IEEE transactions on Smart Grid},
  volume={1},
  number={1},
  pages={28--39},
  year={2010},
  publisher={IEEE}
}

@inproceedings{hines2010topological,
  title={The topological and electrical structure of power grids},
  author={Hines, Paul and Blumsack, Seth and Sanchez, E Cotilla and Barrows, Clayton},
  booktitle={System Sciences (HICSS), 2010 43rd Hawaii International Conference on},
  pages={1--10},
  year={2010},
  organization={IEEE}
}


@article{schultz2014random,
  title={A random growth model for power grids and other spatially embedded infrastructure networks},
  author={Schultz, Paul and Heitzig, Jobst and Kurths, J{\"u}rgen},
  journal={The European Physical Journal Special Topics},
  volume={223},
  number={12},
  pages={2593--2610},
  year={2014},
  publisher={Springer}
}

@article{ghahremani2011online,
  title={Online state estimation of a synchronous generator using unscented Kalman filter from phasor measurements units},
  author={Ghahremani, Esmaeil and Kamwa, Innocent},
  journal={IEEE Transactions on Energy Conversion},
  volume={26},
  number={4},
  pages={1099--1108},
  year={2011},
  publisher={IEEE}
}

@article{ghahremani2016local,
  title={Local and wide-area pmu-based decentralized dynamic state estimation in multi-machine power systems},
  author={Ghahremani, Esmaeil and Kamwa, Innocent},
  journal={IEEE Transactions on Power Systems},
  volume={31},
  number={1},
  pages={547--562},
  year={2016},
  publisher={IEEE}
}

@article{ghahremani2011dynamic,
  title={Dynamic state estimation in power system by applying the extended Kalman filter with unknown inputs to phasor measurements},
  author={Ghahremani, Esmaeil and Kamwa, Innocent},
  journal={IEEE Transactions on Power Systems},
  volume={26},
  number={4},
  pages={2556--2566},
  year={2011},
  publisher={IEEE}
}

@article{wang2012alternative,
  title={An alternative method for power system dynamic state estimation based on unscented transform},
  author={Wang, Shaobu and Gao, Wenzhong and Meliopoulos, AP Sakis},
  journal={IEEE Transactions on Power Systems},
  volume={27},
  number={2},
  pages={942--950},
  year={2012},
  publisher={IEEE}
}

@article{qing2015decentralized,
  title={Decentralized unscented Kalman filter based on a consensus algorithm for multi-area dynamic state estimation in power systems},
  author={Qing, Xiangyun and Karimi, Hamid Reza and Niu, Yugang and Wang, Xingyu},
  journal={International Journal of Electrical Power \& Energy Systems},
  volume={65},
  pages={26--33},
  year={2015},
  publisher={Elsevier}
}

@article{rigatos2013distributed,
  title={A distributed state estimation approach to condition monitoring of nonlinear electric power systems},
  author={Rigatos, G and Siano, P and Zervos, N},
  journal={Asian Journal of Control},
  volume={15},
  number={3},
  pages={849--860},
  year={2013},
  publisher={Wiley Online Library}
}

@article{yang2013false,
  title={On false data injection attacks against Kalman filtering in power system dynamic state estimation},
  author={Yang, Qingyu and Chang, Liguo and Yu, Wei},
  journal={Security and Communication Networks},
  year={2013},
  publisher={Wiley Online Library}
}

@incollection{witczak2014unknown,
  title={Unknown Input Observers and Filters},
  author={Witczak, Marcin},
  booktitle={Fault Diagnosis and Fault-Tolerant Control Strategies for Non-Linear Systems},
  pages={19--56},
  year={2014},
  publisher={Springer}
}

@article{bolandhemmat2012solution,
  title={A Solution to the State Estimation Problem of Systems with Unknown Inputs},
  author={Bolandhemmat, Hamidreza and Clark, Christopher and Golnaraghi, Farid},
  journal={Recent Patents on Mechanical Engineering},
  volume={5},
  number={2},
  pages={102--112},
  year={2012},
  publisher={Bentham Science Publishers}
}

@article{yang1988observers,
  title={Observers for linear systems with unknown inputs},
  author={Yang, Fuyu and Wilde, Richard W},
  journal={IEEE transactions on automatic control},
  volume={33},
  number={7},
  pages={677--681},
  year={1988},
  publisher={Institute of Electrical and Electronics Engineers}
}

@inproceedings{amini2015dynamic,
  title={Dynamic load altering attacks in smart grid},
  author={Amini, Sajjad and Mohsenian-Rad, Hamed and Pasqualetti, Fabio},
  booktitle={Innovative Smart Grid Technologies Conference (ISGT), 2015 IEEE Power \& Energy Society},
  pages={1--5},
  year={2015},
  organization={IEEE}
}

@inproceedings{amini2015detecting,
  title={Detecting dynamic load altering attacks: A data-driven time-frequency analysis},
  author={Amini, Sajjad and Pasqualetti, Fabio and Mohsenian-Rad, Hamed},
  booktitle={2015 IEEE International Conference on Smart Grid Communications (SmartGridComm)},
  pages={503--508},
  year={2015},
  organization={IEEE}
}

@article{fang2012smart,
  title={Smart grid—The new and improved power grid: A survey},
  author={Fang, Xi and Misra, Satyajayant and Xue, Guoliang and Yang, Dejun},
  journal={IEEE communications surveys \& tutorials},
  volume={14},
  number={4},
  pages={944--980},
  year={2012},
  publisher={IEEE}
}

@article{wang2009cascade,
  title={Cascade-based attack vulnerability on the US power grid},
  author={Wang, Jian-Wei and Rong, Li-Li},
  journal={Safety Science},
  volume={47},
  number={10},
  pages={1332--1336},
  year={2009},
  publisher={Elsevier}
}

@article{rosas2007topological,
  title={Topological vulnerability of the European power grid under errors and attacks},
  author={Rosas-Casals, Marti and Valverde, Sergi and Sol{\'e}, Ricard V},
  journal={International Journal of Bifurcation and Chaos},
  volume={17},
  number={07},
  pages={2465--2475},
  year={2007},
  publisher={World Scientific}
}

@article{albert2004structural,
  title={Structural vulnerability of the North American power grid},
  author={Albert, R{\'e}ka and Albert, Istv{\'a}n and Nakarado, Gary L},
  journal={Physical review E},
  volume={69},
  number={2},
  pages={025103},
  year={2004},
  publisher={APS}
}
\end{filecontents}
\immediate\write18{bibtex paper}

%%%%%%%%%%%%%%%%%%%%%%%%%%%%%%%%%%%%%%%%%%%%%%%%%%%%%%%%%%%%%%%%%%%%%%%%%%%%%%%%

\begin{document}

\begin{enumerate}
\item
Filtering theory
\begin{enumerate}
\item
The standard reference for the Unscented Kalman Filter and joint/dual estimation. \cite{wan2000unscented}
When the model is partially linear (as in our case), we can use reduced sigma point filtering to improve speed with no loss in accuracy.
\cite{morelande2006reduced}
We can also improve speed by using a low rank approximation of the covariance matrix.
\cite{padilla2010adaptive}

\item
There exist good filtering methods for designing Unknown Input Observers (UIOs).
\cite{yang1988observers,bolandhemmat2012solution,witczak2014unknown}
Using one of these methods, we can estimate the control input for our system,
then apply the previous techniques to detect attacks.
The UIO technique introduces extra overhead steps, however.
My technique estimates the attack directly without needing to first estimate the control input.
\end{enumerate}

\item
Filtering to estimate states in a power grid
\begin{enumerate}
\item
All of these papers are for estimating the parameters of a single generator connected to an idealized powergrid.
This paper claims that the UKF is better than the EKF for estimating the rotor angle and speed in synchronous generators.
The authors use a fourth-order nonlinear model of a single motor generator with an infinite bus.
No joint estimation is used, and the system is not under attack.
\cite{ghahremani2011online}
This paper uses the EKF-UI algorithm to jointly estimate the parameters of the motor controller and bus loads.
The original system is a fourth-order nonlinear system of the single motor generator.
The system is assumed to be not under attack.
\cite{ghahremani2011dynamic}
Another EKF-UI paper by the same authors.
\cite{ghahremani2016local}

\item
This paper claims that current WAMS monitoring systems of power grids require calculation of a large Jacobian matrix.
They propose using the UKF to avoid the need for a Jacobian, both improving runtime and accuracy.
\cite{wang2012alternative}

\item
This paper claims that using the UKF to monitor realworld power systems is too slow.
They identify two bottlenecks to a centralized monitor.
First, there is a lot of communication required from every node to the monitor.
Second, the monitor must do a lot of computation.
They solve both problems by partitioning the powergrid into disjoint smaller grids,
filtering each grid independently,
and carefully communicating the results between the smaller grids.
A similar technique might be useful to speed up my solution,
but this is a different idea than I had.
\cite{qing2015decentralized}

This paper also does a distributed analysis of realworld power systems as above.
This paper is more similar to our paper because they are specifically looking to detect faults.
They detect faults in a different way.
We have a specific model of the way faults are generated that we are looking to detect by adding new states to the system.
They do not have a specific model, and instead apply a function to the existing states in an attempt to detect faults.
\cite{rigatos2013distributed}
\end{enumerate}

\item
Methods for generating random power grids
\begin{enumerate}
\item
This paper demonstrates that a number of popular models for generating random graphs do not exhibit the topological and electrical properties of powergrids.
\cite{hines2010topological}

\item
The method is relatively simple and tries to directly model powergrid statistics.
Generates random power grids with the properties of real powergrids up to about 300 buses.
\cite{wang2010generating}

\item
This paper has a more complex method that is suitable for any infrastructure, not just powergrids.
The idea is to model how infrastructure actually develops over time due to realworld constraints.
The paper claims to accurately model nation-sized powergrids (more than 1000 buses).
\cite{schultz2014random}
\end{enumerate}

\item
Background info
\begin{enumerate}
\item Smart grid survey
\cite{fang2012smart}

\item
Types of grid attacks
\begin{enumerate}
\item
False data injection attack
\cite{yang2013false}

\item
Dynamic load altering attacks
\cite{amini2015dynamic,amini2015detecting}

\item
Cascade based attacks
\cite{wang2009cascade,rosas2007topological,albert2004structural}
\end{enumerate}
\end{enumerate}
\end{enumerate}

\bibliographystyle{plain}
\bibliography{paper}
\end{document}

